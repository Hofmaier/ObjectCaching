\section{Timon Brüllmann}

In der heutigen Informatik sind verteilte Software Systeme ein sehr interessanter und komplexer Bereich. Mit dieses Themengebiet beschäftigen wir uns in den letzten drei Monate sehr intensiv. Es galt heraus zufinden wie sich die Leistung ein verteiltes System mit eingebauten Cache gegenüber einem ohne Cache-Funktionalität verhält. Während der gesamten Dauer der Arbeit tauchten immer wieder Herausforderungen auf, die es zu lösen galt. Genau diese Probleme bereitetn mir Spass und verlange von uns allen einen grossen Einsatz.


Zu Begin der Studienarbeit mussten wir uns in dieses grosse Themengebiet einlesen und einarbeiten. Am Anfang der Studienarbeit waren die Ziele der Arbeit für uns noch nicht klar definiert gewesen, so das wir zuerst die Ziele, welche wir erreichen wollten definieren mussten. In dieser Phase des Projekts gab es einige schwierige Diskussionen, da jeder von uns eine andere Vorstellung darüber hatte, was man in diesem Projekt ereichen wollte. Nachdem die Ziele klar definiert waren, verliefen die darauf folgenden Wochen innerhalb des Teams als auch im Bezug zur Arbeit sehr gut.  


Die Zusammenarbeit mit Lukas und Raphael verlief grösstenteils positiv, wir haben uns gegenseitig gut ergänzt, woduch die Arbeit in grossen Schritte voran ging und schlussendlich auch erfolgreich abgeschlossen werden konnte.


\chapter{Einleitung}

Die Idee ein RMI-System mit integriertem Client-Cache zu entwickeln, ist bei der Arbeit an Mercury entstanden. Mercury ist ein Datenverarbeitungssystem. Clients können sich an einen Server anmelden und Methoden auf Objekten ausführen, die auf dem Server vorhanden sind. 

Führen mehrere Clients Methoden der gemeinsam genutzten Objekte aus, kann es vorkommen, dass Daten unabsichtlich überschrieben werden. Dieser Fehler kann auftreten, weil Clients Daten über Methoden auslesen und diese Daten später in Methoden mit Schreibzugriff weiterverwenden, obwohl die Daten inzwischen veraltet sind.

Ein Ziel dieser Arbeit ist es, ein Konzept zu entwickeln um, diese Fehler zu verhindern. Anhand dieses Konzeptes wird ein Prototyp diese Systems erstellt. Dieses System ist eine vereinfachte Implementation von RMI und enthält zusätzlich Mechanismen, die Lost Update Fehler verhindern. Der Prototyp soll unter anderem zeigen, ob das Konzept funktioniert. 

Ein weiteres Ziel der Arbeit ist die Verbesserung der Performance in einer RMI Umgebung, durch Replikation der Objekte beim Client. Objekte sollen näher beim Prozess, der darauf zugreift, platziert werden, um die Zugriffszeit zu verkürzen. Deshalb wurde in einem zweiten Schritt ein RMI-System entwickelt, welches Objekte bei den Clients lokal speichert. Lesezugriffe können vom Cache verarbeitet werden.

Mit der Replikation von Objekten entstehen neue Probleme. Gibt es mehrere Kopien von Objekten, müssen Änderungen an einer Kopie an die anderen Kopien weitergegeben werden. Dieser Aktualisierungsprozess kann die Performance wiederum mindern, da ein Server oft damit beschäftigt ist, Kopien zu aktualisieren. Es wurde ein Konzept erstellt, wie ein Caching von Objekten in einem RMI-System realisiert werden kann. Um die Funktionalität zu testen und die Performance zu messen wurde das Konzept implementiert. 

Ein weiteres Problem im Zu\-sam\-men\-hang mit Objekt\-caching ist, dass das Sy\-stem mehr Netzwerkbandbreite benötigt, da es mehr Nachrichten versendet, um die Kopien aktuell zu halten.

Um die Vor -und Nachteile der beschriebenen Konzepte zu finden, muss ein Test-Framework geschrieben werden. Dieses Framework soll Zeitmessungen der einzelnen Methodenaufrufe ermöglichen, was Fehler in den Konzepten schonungslos aufdecken wird. Das Framework soll also zeigen, in welchen Anwendungsfällen sich ein System mit einem Cache lohnt und in welchem Anwendungsfall dies purer Ballast wäre.

Weiter soll das Framework so entwickelt werden, dass neue, zu testende Systeme, einfach angehängt werden können, ohne dass Programmcode angepasst werden muss. Das Framework soll für den gesamten Testablauf eines der oben beschriebenen Systemen verantwortlich sein; vom Start des Servers und der Clients, bis hin zum Herunterfahren dieser Komponenten.


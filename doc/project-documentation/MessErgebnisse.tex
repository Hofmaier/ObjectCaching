\chapter{Messergebnisse}

Dieses Kapitel be\-handelt die Mes\-ser\-ge\-bis\-se, welche mit dem zuvor be\-schrie\-benen System erzielt wurden. Dabei wird von verschiedenen Szenarien ausgegangen, welche die jeweiligen Vor- und Nachteile der zwei Systemen zum Ausdruck bringen sollen. \newline
Alle Mes\-ser\-ge\-bnis\-se stam\-men aus einer Labor\-umgebung, mit folgenden Ei\-gen\-schaf\-ten:
\begin{itemize}
\item Alle Testrechner, Server wie Client, besitzen folgende Hardware:
\begin{itemize}
\item Prozessor: Intel Xeon CPU X3450, 8 x 2.67Ghz
\item Speicher: RAM 8GB
\item Netzwerkkarte: Intel 82578DM Gigabit Network Card
\end{itemize}
\item Folgende Betriebssysteme wurden während den Messungen verwendet:
\begin{itemize}
\item Server: Ubuntu, Version 11.10
\item Auf beiden Clients: Fedora Linux, Release 16
\end{itemize}
\item Die Netzwerkinfrastruktur im Laborraum ist kaum belastet, daher nur durch den gebräuchlichen Leerlaufverkehr(STP, ARP usw.)
\item Die Hardware-Ressourcen der Rechner, auf welchen die Applikation getestet wird, sind ausschliesslich nur durch die Applikation belegt und durch einzelne übliche Prozesse vom  Betriebsystem.
\end{itemize}

\section{Zwei schreibende Clients}

Dieser TestCase sieht folgendermassen aus:
\begin{lstlisting}[breaklines=true]
<?xml version='1.0' encoding='UTF-8'?>
<TestRun>
	<TestCase SystemUnderTest="ch.hsr.objectCaching.rmiOnlyClient.RMIonlyClientSystem" ServerSystemUnderTest="ch.hsr.objectCaching.rmiOnlyServer.RMIOnlyServerSystem">
		<Account balance="1"></Account>
		<Scenario id="1">
			<ActionSequence>

				<Increment count ="1000" delay="0" factor="1.1"></Increment>

			</ActionSequence>
		</Scenario>
		<Scenario id="2">
			<ActionSequence>

				<Increment count ="1000" delay="0" factor="1.1"></Increment>

			</ActionSequence>
		</Scenario>
	</TestCase>
</TestRun>
\end{lstlisting}

Beide Clients erhöhen den "Balance"-Wert auf dem Objekt genau je 1000 Mal um 1.1. Sie führen die Operationen ohne eine Verzögerung zwischen der Lese- und der Schreiboperation aus.

\section{Prognose vor der Messung}

Erwartungsgemäss sollten bei diesem Testfall beide Systeme in etwa gleich schnell sein. Der Cache wird sich, wenn überhaupt nur minimal bemerkbar machen. Der Cache verschnellert zwar den Lesezugriff, also steigert er die Performance bei rund 2000 Operationen, er muss aber auch dementsprechend oft durch den Server auf den neusten Stand gebracht werden. Es ist zu erwarten, dass sich der Performancezuwachs und die Einbussen die Wage halten.

\subsection{RMI-Only System}

\subsection{RMI-Only mit Cache System}

\subsection{Fazit}

\section{Ein lesender, ein schreibender Client}

\subsection{RMI-Only System}

\subsection{RMI-Only mit Cache System}

\subsection{Fazit}

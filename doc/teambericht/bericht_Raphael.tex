\section{Raphi}
Obwohl ich noch nie mit den anderen zwei Projektmitarbeitern ein Projekt getätigt habe, funktionierte die Zusammenarbeit über weite Strecken sehr gut. Da aber die Anforderungen des Projektes bei Projektstart nicht genau definiert waren, folgten hitzige Diskussionen, in welchen wir uns über Ziele, Vorgehensweise und Design stritten. Nachdem diese Punkte geklärt waren, verlief die weitere Arbeit auf einer sehr konstruktiven Basis. \newline
Die Arbeit war wie erwähnt am Anfang relativ schwer. Wir wussten nicht genau, was eigentlich von uns verlangt wurde und in welche Richtung wir gehen sollten. Als wir aber einfach mal loslegten, lief es immer besser. Ich war noch nie an einem Projekt in diesem Ausmass beteiligt und konnte dadurch extrem viel profitieren. Da ich in meinem beruflichen Umfeld keine Software schreibe, konnte ich von meinen beiden Projektmitarbeitern viel lernen. \newline
Vorallem die Arbeit mit mehreren Threads und die ganze parallele Programmierung waren für mich extrem spannend. Wir haben dieses Tätigkeitsgebiet im Studium nur mit einem Modul gestreift. Nun habe ich gelernt, wie man damit umgeht und solche Software entwickelt. \newline
Die Motivation diese Arbeit zu schreiben, war natürlich in erster Linie, dass wir um unser Studium erfolgreich abschliessen zu können, diese Arbeit schreiben müssen. Ich habe mich aber für dieses Thema entschieden, da ich etwas tun wollte, was ich nach meinem Studium in der Wirtschaft nicht mehr tun kann. Wer bezahlt schon einen Mitarbeiter, welcher ein Konzept schreibt, dies testet, dann wieder verwirft, etwas anderes baut usw.? Deshalb war diese Semesterarbeit eine wunderbare Gelegenheit um dies einmal zu tun.


\section{Testing Framework}
\label{sec:testing Framework}
Im folgenden Kapitel werden wir auf die einzelnen Komponenten des von uns entwickelten Test Frameworks eingehen und Ihre Funktionsweise erläutern. Parallel zur Entwicklung des RMIOnly- Systems mit Concurrency Control wurde ein Testframework entwickelt. Die Aufgabe des Testframework lag darin, die verschiedenen Prototypen, welche im Laufe dieser Semesterarbeit entwickelt wurden, mit Hilfe diese Framework einheitlich testen zu lassen.


\subsection{Konzept}
Das Framework lässt sich über eine Konfigurationsdatei konfigurieren und kann Testfälle die in einer XML Datenstruktur vorliegen intepretieren und daraus die einzelnen Testszenarien generieren, welche dann auf die verfügbaren Testframework Clients verteilen werden. Hat ein Testframework Client ein Szenario auf einem ClientSystemUnderTest abgearbeitet, wird dieses zur Auswertung zurück an den Server geschickt, dort werden die Messresultate jedes Szenario ausgewertet und abgespeichert. Dadurch lassen sich die Resultate der verschieden System zu einem späteren Zeitpunkt vergleichen.

\begin{itemize}
\item Labumgebung \/ Client/Server
\end{itemize}


\subsection{Server}
\label{sec:test-FW Server}
\begin{itemize}
\item Szenario definieren via XML (mit Beispiel?)
\item Konfiguration mit Property Datei
\item Resultate \& Auswertung
\item Vorbereitungs Task via Java RMI, Szenarien laden und verteilen
\item Für die Beweisführung das Lost Updates nicht mehr möglich sind, musste ein zusätzlicher Listener eingebaut der die event protokoliert.
\end{itemize}

\subsection{Testframework Client}
\label{sec:test-FW Client}
Diesem Kapitel beschreibt die Aufgaben des Testframework Client und deren Umsetzung. Der Framework Client lässt sich direkt vom Framework Server aus konfigurieren und steuern. Der Testframework Clients besitzt zwei Aufgaben, als erstes erzeugt und konfiguriert er das benötigte ClientSystemUnderTest. Zweitens muss er auf einem ClientSystemUnderTest eine gegebens Szenerio abarbeiten können. Die Realisierung des Testframeworks über das Client/Server Konzepts führte zu eine schlanke Lösung Lösung, welche sich jederzeit erweitern liesse.


\subsubsection{ClientController}
\label{sec:clientController}
Die Kommunikation zwischen den Testframework Clients und dem Server haben wir über Java RMI realisiert. Der Kommunikationkanal wird für das konfigurieren der ClientController und das sammeln der Resultate nach einem Testdurchlauf genutzt. Beim Start eines Testframework Clients wird ein ClientController instanziiert und diese wird dann zur Java RMI Runtime exportiert. Die veröffentlich das ClientController- Objekts erledigt die RMI Runtime im Hindergrund, ist das Objekt erfolgreich registriert, lässt es sich der Testframework Client über Remote Method Invocation vom Server kontrollieren.


Folgendes Interface wird vom ClientController implementiert:
\begin{lstlisting}	
public void initialize(Scenario scenario, Configuration configuration) throws RemoteException;
public void startTest() throws RemoteException;
public void shutdown() throws RemoteException;
\end{lstlisting}
Über dieses Interface lässt sich das gewünschte ClientSystemUnderTest instanziieren und der Testframework Client konfigurieren. Zu Beginn wollten wir die Konfigurationswerte als Parameter übergeben, wir merkten jedoch schnell das es sinnvoller wäre einen Configuration Typ zu erstellen der als reines Data Transfer Objekt dient und alle Werte für die Konfiguration des CUT sowie der Kommunikation zwischen dem Testframework Server und dem Testframework Client kapselt. Des weitern sind im Configuration Objekt alle nötigen Daten gespeichert um erfolgreich eine RMI- Verbindung zum Server aufzubauen. Es wird weiter ein neues TestClient-Objekt erstellt. Der TestClient wird mit dem gegeben Szenario initialisiert. Sind alle Vorbereitungsschritte abgeschlossen, wird der Server über die erfolgreiche initialisierung des CUT und des ClientController informiert. Der Server startet den Testdurchlauf parallel über die Methode startTest() auf allen ClientController, wenn sich zuvor alle ClientController als bereit für einen Testdurchlauf gemeldet haben. Sind alle Aktionen des gegeben Szenarios abgearbeitet, wird das gesamte Szenario mit allen Messresultaten zurück an den Server geschickt. Die ganze Auswertung der Resultate wird vom Server übernommen. 

\paragraph{JVM Status} 
Der Testframework Client sowie das CUT laufen in derselben Java Virtual Machine um für jeden Testcase dieselben Systemvoraussetzungen zu gewährleisten, haben wir uns dazu entschlossen bei jedem Testcase das ganze System neu zu starten. Über die Methode shutdown() lassen lässt sich der Testframework Client sowie der CUT herunterfahren.

\subsubsection{Action}
\label{sec:action}
Das Szenario beinhaltet eine Menge von Aktionen die auf ein Account Objekt ausgeführt werden.  Es lassen sich leicht neue Aktionen definieren und implementieren, da wir das Command-Pattern genutzt haben. Action ist die abstrakte Klasse welche ein Result Objekt instanziiert. Dieses Result Objekt beinhaltet die Messdaten, die für diese Aktion gesammelt wurden. Bei der Ausführung des Scenarios nimmt der TestClient die Liste von Action aus dem Szenario und führt auf jeder Action execute() aus.
 
\subsubsection{Increment Action}
\label{sec:incrementAction}
Für unsere Testumgebung benötigten wir nur eine KontoErhöhungs- Aktion die den aktuellen Kontostand holt und diesen mit dem gegeben Faktor multipliziert und den neuen Kontostand zurück in den Account schreibt. Zusätzlich war es nötig, dass zwischen der der getBalance() und der setBalance() eine definierbare Zeitdauer gewartet werden kann. Dies ist nötig damit sich leicht ein Konflikt erzeugen lässt und so gezeigt werden kann das  dieser serverseitig zu keinem Lost-Updates führt.

\subsubsection{Result}
\label{sec:result}
Die Zeitmessung einer Aktion wir über den in Java eingebaute Zeitmessung System.nanoTime() aufgezeichnet. Die Result Klasse bietet eine startTimeMeasurement Methode die ein neues TimeRecord Objekt erzeugt und mit der Zeitmessung beginnt, diese Methode verlangt eine BasicAction Typ der als Enum realisiert wurde(Erklärung warum nötig unter Resultate). Über stopTimeMeasurement() lässt sich die aktuelle Messung beenden und der aktuelle TimeRecord wird abgelegt. Da innerhalb der execute Methode mehrere setBalance() oder getBalance() möglich sind, lassen sich mehrere TimeRecords pro Result Typ erfassen.


\subsubsection{Test Client}
\label{sec:testclient}
Die Klasse TestClient ist die Schnittstelle zwischen dem Testframework und dem ClientUnderTestSystem, welches getestet werden soll. Dieser Klasse kann im Konstruktor ein Objekt übergeben werden, welches das ClientSystemUnderTest Interface implementiert. Dieses Interface bietet folgende Methoden an:
\begin{lstlisting}
public AccountService getAccountService();
public void setServerSocketAdress(InetSocketAddress socketAdress);
public void shutdown();
\end{lstlisting}	
Über die getAccountService() Methode kann auf eine Liste von Account Objekten zugegriffen werden. Die Methode runScenario führt das gegebene Scenario aus hierfür wird auf dem Action Objekt die execute() Methode aufgerufen. Diese Methode verlangt ein Account Objekt auf welchem es die entsprechende Aktion ausführt. Der TestClient ist ebenfalls für die ordnungsgemässe Beendigung des ClientUnderTestSystem verantwortlich.


\subsection{TODO}
\begin{itemize}
\item RMI
\item Szenario -> Kommand-Pattern
\item Zeitmessung system.nanotime()
\item Client herunterfahren
\item Testclient Interface für ClientSystemUnderTest Interface
\item VM Profiler (Abfallobjekte anschauen)
\item Lost Update wird auf Server verhindert, Client muss entsprechen auf Exeption reagieren, neuer Versuch starten
\end{itemize}



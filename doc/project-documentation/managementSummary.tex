\chapter*{Management Summary}

\section*{Ausgangslage}
Die Firma staila technologies entwickelt ein neuartigs Datenverarbeitungssystem mit dem Namen Mercury. Die höchste Priorität bei der Entwicklung kommt der Performance des Systems zu. Um dies zu erreichen wurden neue Ideen und Konzepte entworfen, welche die heutigen, üblichen Vorgehensweisen verlassen und einen neuen Weg gehen.

Im Rahmen der Entwicklungsarbeit an Mercury, soll mit dieser Arbeit ein weiteres Puzzleteil hinzugefügt werden. Um eine möglichst performante Kommunikation zwischen Client und Server in Mercury herstellen zu können, gibt es verschiedenste Algorithmen, welche für verteilte Systeme geeignet sind. Das Ziel dieser Arbeit ist für viel versprechende Ansätze einen Prototypen zu entwickeln und diesen dann zu testen. Aus den verschiedensten Tests soll dann ersichtlich werden, welcher Algorithmus sich für Mercury am besten eignet.

\section*{Vorgehen / Technologien}
Anfangs wurde ein einfaches, zuverlässiges System entwickelt, um Referenzwerte für die folgenden Zeitmessungen zu erlangen. In einem nächsten Schritt wurde ein Prototyp gebaut, welcher einen lokalen Cache besitzt. Mit diesem Prototyp wurden wiederum Zeitmessungen durchgeführt und die Werte wurden miteinander verglichen.

Parallel zur Entwicklung der oben genannten Prototypen, wurde ein Framework entwickelt, welches die Systeme testet. Es wurde dabei darauf geachtet, dass die Prototypen ohne Anpassung von Programmcode an das Framework angehängt werden können. Das Framework ist für die Zeitmessungen und die Ausgabe der Resultate, sowie für viele weitere Aufgaben, verantwortlich.

Alle Prototypen und das Framework wurden in Java entwickelt. Da keine andere Programmiersprache gewünscht wurde und wir am meisten Erfahrung mit dieser Sprache gesammelt haben, war dies eine logische Wahl.
\section*{Ergebnisse}
Das Ziel dieser Arbeit, zwei Prototypen und ein Framework zu entwickln, wurde erreicht. Die Probleme, welche sich während der Entwicklungsarbeit stellten wurden in den folgenden Seiten dokumentiert. Weiter wurden die Entscheidungen und die Gründe, warum genau dieser Weg eingeschlagen wurde, genaustens dokumentiert.

Am Ender dieser Arbeit werden die Messergebnisse ausgewiesen. Aus den Messergebnissen werden Schlüsse und Interpretationen gezogen. Zusätzlich wurde das Kapitel mit einzelnen Grafiken versehen, um die Auswertung der Daten ansehlicher und besser verständlich zu machen.

Diese Arbeit ist als Grundlage für zukünftige Entscheidungen bezüglich des Einsatzes eines Caches zu sehen. Es können Antworten zu den gängigsten Problemen, welche sich bei der Entwicklung eines solchen Systems stellen, gefunden werden. Weiter kann aufgrund der aufgeführten Messdaten abgeschätzt werden, ob sich die Einbindung und der damit verbundene Aufwand für die Entwicklung eines Caches, lohnt.

\section{Testing Framework}
\label{sec:testing Framework}
\begin{itemize}
\item Warum das FW
\item Idee verschiedene Clients testen
\end{itemize}

\subsection{Konzept}
Parallel zur Entwicklung des RMI mit Concurrency Control wurde ein Testframework entwickelt. Die Aufgabe des Testframework ist, dass die verschiedenen Prototypen mit Hilfe des Frameworks einheitlich getestet werden können. Das Framework kann verschiedene Testszenarien abarbeiten und speichert die Resultate für jedes Szenario ab. blabla


\subsection{Realisierung des Test Frameworks}
\label{sec:real test-FW}
Sequenzdiagramm von Ablauf init(), start() setResult(), shutdown()

\subsection{Server}
\label{sec:test-FW Server}
\begin{itemize}
\item Aufbau / Infra
\item Szenario definieren via XML (mit Beispiel?)
\item Konfiguration mit Property Datei
\item Resultate \& Auswertung
\item Shell, Deployment usw.
\item Vorbereitungs Task via Java RMI, Szenarien laden und verteilen
\item Für die Beweisführung das Lost Updates nicht mehr möglich sind, musste ein zusätzlicher Listener eingebaut der die event protokoliert.

\end{itemize}

\subsection{Client}
\label{sec:test-FW Client}
\begin{itemize}
\item RMI
\item Szenario -> Kommand-Pattern
\item Zeitmessung system.nanotime()
\item Client herunterfahren
\item Testclient Interface für ClientSystemUnderTest Interface
\item VM Profiler (Abfallobjekte anschauen)
\item Lost Update wird auf Server verhindert, Client muss entsprechen auf Exeption reagieren, neuer Versuch starten
\end{itemize}
\chapter{Einleitung}

Die Idee ein RMI-System mit integriertem Client-Cache zu entwickeln, ist bei der Arbeit an Mercury entstanden. Mercury ist ein Datenverarbeitungssystem. Clients können sich an einen Server anmelden und Methoden auf Objekte ausführen, die auf dem Server vorhanden sind. 

Ein Ziel dieser Arbeit ist es ein Konzept zu entwickeln um
diese Fehler zu verhindern. Anhand dieses Konzeptes wird ein Prototyp
diese Systems erstellt. Dieses System ist eine veeinfachte
Implementation von RMI und enthält zusätzlich Mechanismen, die Lost
Update Fehler verhindern. Der Prototyp soll unter anderem zeigen, ob
das Konzept funktioniert. 

Ein weiteres Ziel dieses RMI-Systems ist die Verbesserung der Performance durch Replikation der Objekte beim Client. Objekte sollen näher beim Prozess, der darauf zugreift platziert werden, um die Zugriffszeit zu verkürzen.

Mit der Replikation von Objekten entstehen neue Probleme. Gibt es mehrere Kopien von Objekten müssen Änderungen an einer Kopie an die anderen Kopien weitergegeben werden. Dieser Aktualisierungsprozess kann die Performance wiederum mindern, da ein Server oft damit beschäftigt ist, Kopien zu aktualisieren.

Ein weiteres Problem im Zusammenhang mit Objektcaching, ist dass das System mehr Netzwerkbandbreite benötigt, da es mehr Nachrichten versendet, um die Kopien aktuell zu halten.

Ein weiteres Ziel der Arbeit ist die Verbesserung der Performance in einer RMI Umgebung durch Replikation der Objekte beim Client. Objekte sollen näher beim Prozess, der darauf zugreift platziert werden, um die Zugriffszeit zu verkürzen.

Mit der Replikation von Objekten entstehen neue Probleme. Gibt es mehrere Kopien von Objekten müssen Änderungen an einer Kopie an die anderen Kopien weitergegeben werden. Dieser Aktualisierungsprozess kann die Performance wiederum mindern, da ein Server oft damit beschäftigt ist, Kopien zu aktualisieren. Es wurde ein Konzept erstellt, wie ein Caching von Objekten in einem RMI-System realisiert werden kann. Um die Funktionalität zu testen und die Performance zu messen wurde das Konzept implementiert. 


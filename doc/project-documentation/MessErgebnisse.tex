\chapter{Messergebnisse}

Dieses Kapitel behandelt die Messergebisse, welche mit dem zuvor beschriebenen System erzielt wurden. Dabei wird von verschiedenen Szenarien ausgegangen, welche die jeweiligen Vor- und Nachteile der zwei Systemen zum Ausdruck bringen sollen. \newline

\section{Laborumgebung}

Alle Messergebnisse stammen aus einer Laborumgebung, mit folgenden Eigenschaften:
\begin{itemize}
\item Alle Testrechner, Server wie Client, besitzen folgende Hardware:
\begin{itemize}
\item Prozessor: Intel Xeon CPU X3450, 8 x 2.67Ghz
\item Speicher: RAM 8GB
\item Netzwerkkarte: Intel 82578DM Gigabit Network Card
\end{itemize}
\item Folgende Betriebssysteme wurden während den Messungen verwendet:
\begin{itemize}
\item Server: Ubuntu, Version 11.10
\item Auf beiden Clients: Fedora Linux, Release 16
\end{itemize}
\item Die Netzwerkinfrastruktur im Laborraum ist kaum belastet, daher nur durch den gebräuchlichen Leerlaufverkehr(STP, ARP usw.)
\item Die Hardware-Ressourcen der Rechner, auf welchen die Applikation getestet wird, sind ausschliesslich nur durch die Applikation belegt und durch einzelne, übliche Prozesse vom  Betriebsystem.
\end{itemize}

\section{Testvorgaben}
Die in dieser Arbeit ausgewiesenen Zahlen wurden alle in der Laborumgebung gemessen. Dabei wurde für jeden TestCase die Anzahl der Clients schrittweise von zwei bis acht Clients erhöht. Jeder Testlauf mit der gleichen Anzahl Clients, wurde drei Mal durchgeführt, um einen exakten Mittelwert der Ergebnisse zu bekommen. Drei Testdurchläufe pro Scenario und Clientanzahl, wurden aufgrund der sehr nahe liegenden Ergbnisse der Testdurchläufe als genügen eingestuft. Wären die einzelnen Ergbnisse der Durchläufe extrem unterschiedlich, hätten fünf oder mehr Durchläufe stattfinden müssen.

\section{Testergebnisse}
In den folgenden Kapiteln werden die getesteten Scenarios und deren Ergebnisse gezeigt. 

\subsection{Nur lesende Clients}
\subsubsection{Szenario Code}
Dieser TestCase sieht folgendermassen aus:
\begin{lstlisting}
<?xml version='1.0' encoding='UTF-8'?>
<TestRun>
	<TestCase
		ClientSystemUnderTest="ch.hsr.objectCaching.rmiWithCacheClient.RMIwithCacheClientSystem"
		ServerSystemUnderTest="ch.hsr.objectCaching.rmiWithCacheServer.RMIWithCacheServerSystem">
		<Account balance="1"></Account>
		<Scenario id="1">
			<ActionSequence>
				<Read count="250"></Read>
			</ActionSequence>
		</Scenario>
	</TestCase>
</TestRun>
\end{lstlisting}

\subsubsection{Szenariobeschreibung}
Bei diesem Testcase führen alle Clients die gleichen Operationen aus. Alle Clients lesen das Objekt auf dem Server genau 250 Mal. Das Account-Objekt wird mit dem Wert "1" initialisiert, welcher bis zum Schluss unverändert bleiben wird. \newline
Zu erwarten ist ein enormer Geschwindigkeitsvorteil des Cache-Systems. Da bei diesem Testcase keine Schreiboperationen getätigt werden, wird ein Update der Clients nie nötig sein. Dadurch muss der Cache nur beim ersten Lesezugriff aktualisiert werden und die weiteren Lesezugriffe können lokal auf dem Rechner abgewickelt werden. Daher scheint es nur logisch, dass das Cache-System um einiges schneller ist.

\subsubsection{Ergebnisse RMI-Only System}

\subsubsection{Ergebnisse RMI-Only mit Cache System}



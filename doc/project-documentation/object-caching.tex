
\section{Object Caching}
\label{sec:object-caching}

\subsection{Replica Placement}
\label{sec:replica-management}

Ziel des Object Caching ist es die Zugriffszeit auf Objekte für Client zu verringern. Deshalb macht es Sinn, dass Repliken vom Client initiiert werden. Die Alternative wäre dass der Server die Repliken initiiert. Werden die Replikas vom Client initiiert spricht man von Client Caches. Client Caches verbessern die Zugriffszeit der Clients auf die Daten.

In unserem RMIwithObjectCaching-System besitzt jeder Client einen lokalen Cache auf derselben Maschine. Der Cache ist nicht in der Grösse limitiert, da die Testcases keine Szenarien mit vielen Objekten vorsehen.

\subsection{Update Distribution}
\label{sec:update-distribution}

Werden Kopien von Objekten in einem lokalen Cache angelegt, müssen diese Kopien aktualisiert werden. Um dies zu realisieren gibt es mehrere Möglichkeiten.

\subsubsection{Invalidation versus Data Transfer}
\label{sec:inval-vers-data}

\begin{description}
\item[Invalidation protocol] Bei einem Invalidation protocol werden nur Meldungen an die lokalen Caches gesendet, die dem Cache mitteilen, dass ein Objekt nicht mehr aktuell ist. Der Vorteil dieser Möglichkeit ist, das eine Invalidierungsmeldung nur beim ersten Write versendet werden muss. Ausserdem müssen keine Objektdaten übertragen werden. Das Verfahren spart also Bandbreite. Ein invalidation protocol macht ist sinnvoll, wenn das read-to-write-Verhältnis klein ist.
\item[Transfer Data] Der zweite Ansatz ist bei jedem Update die kompletten Daten eines Objektes an die lokalen Caches zu versenden. In diesem Fall kann der Client immer aus dem lokalen Cache lesen. Dieser Ansatz macht Sinn bei einem hohen read-to-write Verhältnis.
\end{description}

\subsubsection{Pull versus Push}
\label{sec:pull-versus-push}

Die Verantwortung, der Cache Aktualisierung kann entweder beim Server oder beim Client liegen. Man unterscheided zwischen push-based und pull-based Protokollen.

\begin{description}
\item[push-based] Der Server sendet dem Client Updates ohne, dass der Client diese anfordert. Der Server muss Buch darüber führen, in welchen Caches Kopien aller Objekte vorhanden sind.
\item[pull-based] Clients überprüfen bei jedem Zugriff, ob die Daten im Cache aktuell sind. Wenn nicht müssen die Daten neu angefordert werden. Das macht den Zugriff auf Daten langsamer. Dafür muss sich der Server nicht darum kümmern.
\end{description}

In unsererem Cache werden Aktualisierungen durch push-based updated realisiert, da wir die Zugriffzeit für die Clients erhöhen wollen und sich die Anzahl der Clients in Grenzen hält. Der Server kann sich merken, welcher Cache welche Objekte enthält.

In unserem System sind alle Objekte zentral auf dem Server abgespeichert. Das Objekt auf dem Server ist das Referenzobjekt. Alle Clients sollen die Updateoperationen in der Reihenfolge sehen, wie sie beim Server eintreffen. Ist jedem Objekt eine Maschine für die Koordination zugeordnet, nennt man das Protokol primary-based. Die Maschine, die das Objekt verwaltet nennt man primary. Ist der primary eine fixer Server werden alle Updateoperationen remote an diesen Server gesendet. In diesem Fall spricht man von Remote-Write Protocol. Die Concurrency Control bleibt dabei die selbe. Der inkrementiert die Version für jedes Objekt nachdem er die die Update Methode ausgeführt hat. Erhält vor dem Update Call einen Updates des Clients wird eine Exception geworfen.

\subsection{Implementierung Object Konsistenz Protokol}
\label{sec:impl-object-kons}

Der Prozess Kontostand erhöhen läuft wie folgt ab.

\begin{enumerate}
\item Client $C1$ ruft \verb+getBalance()+ auf.
\item RMI-Stub führt einen Lookup für das Objekt im Cache aus.
\item Ist das Objekt nicht Cache, fordert es der Cache beim Server an.
\item \verb+getBalance()+ wird auf auf dem Objekt ausgeführt. Der Wert wird an den Stub zurückgegeben und der Stub leitet ihn an den Client weiter.
\item \verb+setBalance()+ wird an den lokalen Cache weitergeleitet. Dieser enthält eine eigene Concurrency Control. Die überprüft, ob das Objekt sich seit dem letzen \verb+getBalance+ nicht verändert hat. Hat es sich verändert, wird eine Exception ausgelöst. Die Methode ändert den lokalen Cache nicht, weil zu diesem Zeitpunkt noch nicht klar ist, ob die Methode ausgeführt werden kann. Ein anderer Client könnte, das Objekt inzwischen verändert haben. Nur der Server kann entscheiden, ob die Methode ausgeführt werden kann.
\item Die \verb+setBalance()+-Methode invalidiert das Objekt. Der Wert im Cache kann noch nicht festgeschrieben werden, da noch nicht sicher ist, ob die Update Methode ausgeführt werden kann. Der Wert wird sich sicher ändern. Entweder aufgrund des lokalen \verb+setBalance()+-Aufruf oder weil ein anderer Client \verb+setBalance()+ aufgerufen hat. Solange das Objekt invalidiert ist, muss es bei einem \verb+getBalance()+-Aufruf neu beim Server angeforder werden. Damit verhindert man eine unnötige Exception.
\item Erhält ein Client Cache ein Update \verb+update()+
\end{enumerate}

\begin{description}
\item [setBalance] Das Object muss im Cache invalidiert werden. Mit ConcurrencyControl kann überprüft werden, ob mit einer alten Version gearbeitet wurde.
\end{description}


\subsubsection{Erwartetet Vorteile}
\label{sec:erwartetet-vorteile}

\begin{itemize}
\item \verb+getBalance()+-Aufrufe werden schneller ausgeführt, da der Aufruf nicht über einen Socket geschickt werden muss, sondern im RAM bleibt, falls das Object im Cache vorhanden ist.
\item Bei \verb+setBalance()+ werden mögliche Inkonsistenzen bereits beim Client festgestellt. Der User erhält schneller Feedback und der Server wird weniger durch Concurrency Control belastet.
\end{itemize}

\subsubsection{Domain Klassen des Client}
\label{sec:domain-klassen}

\begin{description}
\item[Cache] Die Stubs, z.B. AccountStub, fordern beim Cache Objekte an. Der Cache holt die Objekte aus dem Memory oder er fodert sie über Netzwerk beim Server an. Er fordert sie beim Server an, wenn sie noch nicht vorhanden sind oder wenn ein Objekt invalidiert wurde. Der Cache muss bei einem Update, die Objekte im Hauptspeicher aktualisieren.
\item[Concurrency Control] Überprüft ob \verb+getBalance+ und \verb+setBalance()+ ausgeführt werden können, ohne dass Inkonsistenzen entstehen.
\item[Message Manager] Nimmt Messages für Remote-Methodenaufruf und Objekt-Request entgegen und sorgt dafür dass sie an den Server versendet werden und die Antwort darauf an den Absender weitergeleitet werden. Nimmt vom Server Update-Messages entgegen und leitet sie an den Cache weiter. 
\end{description}


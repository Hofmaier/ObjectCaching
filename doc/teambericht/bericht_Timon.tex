\section{Timon Brüllmann}

Eine anspruchsvolle Disziplin der Informatik sind verteilte Software Systeme. Mit diesem Themengebiet beschäftigen wir uns in den letzten drei Monaten sehr intensiv. Es galt heraus zu finden, wie sich die Leistung eines verteilten Systems mit eingebautem Cache gegenüber einem ohne Cache-Funktionalität verhält. Während der gesamten Dauer der Arbeit tauchten immer wieder Herausforderungen auf, die es zu lösen galt. Genau diese Probleme bereiteten mir Spass und verlangen von uns allen einen grossen Einsatz.


Zu Beginn der Studienarbeit mussten wir uns zuerst in dieses Themengebiet einlesen und einarbeiten. Zugleich waren die Ziele der Arbeit für uns noch nicht klar definiert gewesen, so dass wir diese zuerst festlegen mussten. In dieser Phase des Projekts gab es einige schwierige Diskussionen, da jeder von uns eine andere Vorstellung darüber hatte, was man mit dieser Arbeit erreichen wollte. Nachdem die Ziele klar definiert waren, verliefen die darauf folgenden Wochen innerhalb des Teams, als auch im Bezug zum Arbeitsfortschritt sehr gut.  


Die Zusammenarbeit mit Lukas und Raphael verlief grösstenteils positiv, wir haben uns gegenseitig gut ergänzt, wodurch die Arbeit in grossen Schritten voran ging und schlussendlich auch erfolgreich abgeschlossen werden konnte.


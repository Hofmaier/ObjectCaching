\newglossaryentry{lostupdate}
{
  name=Lost Update,
  description={Verlorenes Update bezeichnet in der Informatik einen Fehler, der bei mehreren parallelen Schreibzugriffen auf eine gemeinsam genutzte Information auftreten kann.}
}
\newglossaryentry{RMI}
{
  name=RMI,
  description={RMI heisst "'Remote Method Invocation" und ermöglicht das Aufrufen von Methoden auf anderen Rechnern.}
}
\newglossaryentry{concurrencyControl}
{
  name=Concurrency Control,
  description={Behandelt die korrekte Planung und Durchführung von miteinander in Konflikt stehenden Operationen.}
}
\newglossaryentry{objectCaching}
{
  name=Object-Caching,
  description={Beschreibt das Ablegen von Objekten im lokalen Speicher.}
}
\newglossaryentry{account}
{
  name=Account,
  description={Die in diesem Dokument vorgestellten Konzepte werden alle anhand eines simplen Bankkonto-Objektes beschrieben. Das Bankkonto verfügt über einen Saldo, eine Methode um darauf zu schreiben und eine zweite um den Saldo auszulesen. }
}
\newglossaryentry{interface}
{
  name=Interface,
  description={Eine Schnittstelle (englisch interface) dient in der objektorientierten Programmierung der Vereinbarung gemeinsamer Signaturen von Methoden, die in unterschiedlichen Klassen implementiert werden. Die Schnittstelle gibt dabei an, welche Methoden vorhanden sind oder vorhanden sein müssen.}
}
\newglossaryentry{konsistenz}
{
  name=Konsistenz,
  description={Als Konsistenz bezeichnet man die Korrektheit der gespeicherten Daten nach einem schreibenden Zugriff.}
}
\newglossaryentry{optimisticConcurrency}
{
  name=Optimistic Concurrency,
  description={Bei diesem Verfahren "'hofft"' man, dass keine Konflikte auftreten. Man ist optimistisch.}
}
\newglossaryentry{middleware}
{
  name=Middleware,
  description={Middleware oder Vermittlungssoftware bezeichnet in der Informatik anwendungsneutrale Programme, die so zwischen Anwendungen vermitteln, dass die Komplexität dieser Applikationen und ihre Infrastruktur verborgen werden.}
}
\newglossaryentry{virtualMachine}
{
  name=Virtual Machine,
  description={Die Java Virtual Machine ist der Teil der Java-Laufzeitumgebung (JRE) für Java-Programme, der für die Ausführung des Java-Bytecodes verantwortlich ist.}
}
\newglossaryentry{requestReplyProtocol}
{
  name=Request-Reply Protocol,
  description={Ein Protokoll, welches meist nur ein Datenpacket sendet und dann mit dem Senden weiterer Packete wartet, bis es eine Antwort erhalten hat.}
}
\newglossaryentry{TCP}
{
  name=TCP,
  description={Das Transmission Control Protocol (TCP) ist eine Vereinbarung darüber, auf welche Art und Weise Daten zwischen Computern ausgetauscht werden sollen.}
}
\newglossaryentry{serialisierte}
{
  name=Serialisierte,
  description={Serialisierung wird hauptsächlich für die Persistierung von Objekten in Dateien und für die Übertragung von Objekten über das Netzwerk bei Verteilten Softwaresystemen verwendet.}
}
\newglossaryentry{hashMap}
{
  name=Hashmap,
  description={In der Informatik bezeichnet man eine spezielle Indexstruktur als Hashtabelle bzw. Streuwerttabelle. Als Indexstruktur werden Hashtabellen verwendet um Datenelemente in einer großen Datenmenge aufzufinden.}
}
\newglossaryentry{STP}
{
  name=STP,
  description={Das Spanning Tree Protocol (STP) baut einen Spannbaum zur Vermeidung von Schleifen in redundanten Netzpfaden im LAN, speziell in geswitchten Umgebungen, auf.}
}
\newglossaryentry{ARP}
{
  name=ARP,
  description={Das Address Resolution Protocol (ARP) ist ein Netzwerkprotokoll, das zu einer Netzwerkadresse der Internetschicht die physikalische Adresse (Hardwareadresse) der Netzzugangsschicht ermittelt und diese Zuordnung gegebenenfalls in den so genannten ARP-Tabellen der beteiligten Rechner hinterlegt.}
}

